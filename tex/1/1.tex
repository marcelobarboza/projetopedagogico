\chapter{Apresenta\c{c}\~{a}o}
\label{cap:apresentacao}

\section{Apresenta\c{c}\~{a}o}
\label{sec:apresentacao}

A implantação do curso de Licenciatura em Matemática por esta
Instituição vem no primeiro momento atender as diferentes demandas, em
particular a formação de professores para a área de Matemática e
expandir a quantidade de vagas ofertadas pelo Ensino Público Federal
em especial a partir da promulgação da Lei 11892/2008.

A iniciativa desta Instituição em ofertar o curso de licenciatura tem
o intuito de desenvolver o ensino na região como objetivo imediato,
mas vai além quando pensamos na problematização sobre a educação,
especificamente na área de exatas, vivenciada pelo país.

Todo projeto ao ser construído tem um objetivo, uma visão de mundo, de
homem e de sociedade que visa educar, função estáa qual ele se propõe.
Como futuro educador, o aluno da licenciatura necessita do
conhecimento da área da Matemática, das ciências afins e compreender
sobre a organização e funcionamento do sistema educacional, seus
conflitos, objetivos e problemas, posicionamento filosófico e
sociológico, além do conhecimento sobre o ser humano, sua capacidade e
forma de constituição biológica e cognitiva e o estabelecimento de
relações interpessoais.

Esta proposta para o curso Licenciatura em Matemática preocupa com a
distribuição de carga horária dedicada às diferentes áreas da
Matemática e das ciências afins primando pelo processo de ensino e
aprendizagem, o que permite flexibilizar a formação do egresso,
adequando o curso às novas exigências legais internas e externas.

De tal modo, este Projeto Pedagógico está alicerçado numa concepção de
educação escolar, no qual o processo educacional é entendido como
aprendizagem para a construção histórica e social dos homens,
articulando os elementos que envolvem o agir e o existir humano, sendo
a aprendizagem o núcleo central.

No entanto, a aquisição de um novo pensamento estruturante do fazer
pedagógico só é possível a partir de uma nova concepção de homem, de
sociedade e de mundo e de uma visão do conhecimento como uma
construção histórica e dinâmica, e necessariamente ligada à prática
social.

A proposta de implantação do curso de Licenciatura em Matemática
apresentada neste projeto pedagógico atende aos seguintes requisitos
legais:

\begin{itemize}[label=-]
  \setlength{\itemsep}{0.2cm}
  \setlength{\parskip}{0.2cm}
  \setlength{\parsep}{0.2cm}
\item \textbf{Lei de Diretrizes e Bases da Educação Nacional} --
  LDB9394/1996 -- que estabelece as diretrizes e bases da Educação
  Nacional;
\item \textbf{Parecer CNE/CES 1302/2001 e Resolução CNE/CES 3/2003} --
  estabelece as Diretrizes Curriculares Nacionais para os Cursos de
  Matemática, Bacharelado e Licenciatura, e tem por objetivo `servir
  como orientação para melhorias e transformações na formação' do
  Licenciado em Matemática, bem como `assegurar que os egressos tenham
  sido adequadamente preparados para uma carreira na qual a Matemática
  seja utilizada de modo essencial, assim como para um processo
  contínuo de aprendizagem';
\item \textbf{Resolução CNE/CP 01/2002 alterada pela resolução CNE/CP
    1/2005}, que instituias Diretrizes Curriculares Nacionais paraa
  Formação de Professores da Educação Básica, em nível superior, curso
  de licenciatura, de graduação plena e constitui-se `de um conjunto
  de princípios,fundamentos e procedimentos a serem observados na
  organização institucional e curricular de cada estabelecimento de
  ensino'.  Segundo tal resolução, além de atender o disposto nos
  artigos 12 e 13 da Lei de Diretrizes e Bases -- LDB 9394/96 -- a
  organização curricular de cada instituição deve observar outras
  formas de orientação inerentes à formação para a atividade docente,
  entre as quais o preparo para:
  \begin{enumerate}[label=\roman*.]
  \item o ensino visando \`{a} aprendizagem do aluno;
  \item o acolhimento e o trato da Diversidade;
  \item o exerc\'{i}cio de atividades de enriquecimento cultural;
  \item o aprimoramento em pr\'{a}ticas investigativas;
  \item a elabora\c{c}\~{a}o e a execu\c{c}\~{a}o de projetos de
    desenvolvimento dos conte\'{u}dos curriculares;
  \item o uso de tecnologias da informa\c{c}\~{a}o e
    comunica\c{c}\~{a}o e de metodologias, estrat\'{e}gias e materiais
    de apoio inovadores;
  \item o desenvolvimento de h\'{a}bitos de colabora\c{c}\~{a}o e
    trabalho em equipe;
  \end{enumerate}
\item \textbf{Resolução CNE/CP 02/2002}, que institui a duração e a
  carga horária dos cursos de licenciatura, de graduação plena, de
  formação de professores da Educação Básica em nível superior. Esta,
  em seu artigo 1\textsuperscript{\d o}, estabelece que `a carga
  horária dos cursos' de Formação de Professores da Educação Básica,
  em nível superior, em curso de licenciatura, de graduação plena,
  será efetivada mediante a integralização de, no mínimo, 2800 (duas
  mil e oitocentas) horas, nas quais a articulação teoria-prática
  garanta, nos termos dos seus projetos pedagógicos, as seguintes
  dimensões dos componentes comuns:
  \begin{enumerate}[label=\roman*.]
  \item 400 (quatrocentas) horas de prática como componente
    curricular, vivenciadas ao longo do curso;
  \item 400 (quatrocentas) horas de est\'{a}gio curricular
    supervisionado a partir do in\'{i}cio da segunda metade do curso;
  \item 1800 (hum mil e oitocentas) horas de aulas para os
    conte\'{u}dos curriculares de natureza cient\'{i}fico-cultural;
  \item 200 (duzentas) horas para outras formas de atividades
    acad\^{e}mico-cient\'{i}fico-culturais;
  \end{enumerate}
\item \textbf{Decreto n\textsuperscript{\d o} 5.626/2005} --
  disp\~{o}e sobre a L\'{i}ngua Brasileira de Sinais;
\item \textbf{Resolução CNE/CP n\textsuperscript{\d o} 1/2004}
  (alterada pela Lei n\textsuperscript{\d o} 11.645/2008) --
  estabelece as Diretrizes Curriculares Nacionais para Educação das
  Relações Étnico-Raciais a serem observadas pelas instituições de
  ensino, em especial, por instituições que desenvolvem programas de
  formação inicial e continuada de professores.  Com estudos da
  Históriae Cultura Afro-brasileira e Indígena;
\item \textbf{Lei n\textsuperscript{\d o} 9795/1999 e Decreto
    n\textsuperscript{\d o} 4.281/2002} -- institui a políticaNacional
  de Educação Ambiental. O que se pretende é formar um profissional
  competente, criativo, crítico, que domine os aspectos filosóficos,
  históricos, culturais, políticos, sociais, psicológicos e
  metodológicos queserelacionam com o trabalho do professor, com a
  gestão da escola, com a educação de jovens cidadãos brasileiros e
  com a construção de uma sociedade democrática e includente, buscando
  respostas aos desafios e problemas existentes nas escolas
  brasileiras;
\item \textbf{Sociedade Brasileira de Educa\c{c}\~{a}o Matem\'{a}tica}
  (SBEM) -- Subsídios para a discussão de propostas para os cursos de
  Licenciatura em Matemática, Abril/2002;
\item \textbf{Resolução CNE/CP 02/2015 Diretrizes Curriculares
    Nacionais para a Formação Inicial e Continuada dos Profissionais
    do Magistério da Educação Básica}, que institui a dura\c{c}\~{a}o
  e a carga hora\'{a}ria dos cursos de forma\c{c}\~{a}o de professores
  da Educa\c{c}\~{a}o B\'{a}sica em n\'{i}vel superior. Esta, em seu
  artigo 1\textsuperscript{\d o}, estabelece que a carga hor\'{a}ria
  dos cursos de Forma\c{c}\~{a}o de Professores da Educação Básica, em
  n\'{i}vel superior, em curso de licenciatura, de gradua\c{c}\~{a}o,
  ser\'{a} efetivada mediante a integraliza\c{c}\~{a}o de, no
  m\'{i}nimo, 3.200 (tr\^{e}s mil e duzentas) horas, nas quais a
  articula\c{c}\~{a}o teoria-pr\'{a}tica garanta, nos termos dos seus
  projetos pedag\'{o}gicos, as seguintes dimens\~{o}es dos componentes
  comuns: \\
  \vspace{1cm}\hfill \adjustbox{max totalheight=0.5\textheight}{%
    \begin{minipage}[t]{10cm}
      Os cursos de formação inicial de professores para a educação
      básica em nível superior, em cursos de licenciatura, organizados
      em áreas especializadas, por componente curricular ou por campo
      de conhecimento e/ou interdisciplinar, considerando-se a
      complexidade e multirreferencialidade, estruturam-se por meio da
      garantia de base comum nacional das orientações curriculares,
      constituindo-se de, no mínimo, 3.200 (três mil e duzentas) horas
      de efetivo trabalho acadêmico, o, em cursos com duração de, no
      mínimo, 8 (oito) semestres ou 4 (quatro) anos, compreendendo:
      \begin{enumerate}[label=\alph*.]
      \item 400 (quatrocentas) horas de prática como componente
        curricular, distribuídas ao longo do processo formativo;
      \item 400 (quatrocentas) horas dedicadas ao estágio
        supervisionado, na área de formação e atuação na educação
        básica, contemplando também outras áreas específicas, se for o
        caso, conforme o projeto de curso da instituição;
      \item 2.200 (duas mil e duzentas) horas dedicadas às atividades
        formativas estruturadas pelos núcleos I e II, conforme o
        projeto de curso da instituição;
      \item 200 (duzentas) horas de atividades teórico-práticas de
        aprofundamento em áreas específicas de interesse dos
        estudantes, como definido no núcleo III, por meio da iniciação
        científica, da iniciação à docência, da extensão e da
        monitoria, entre outras, conforme o projeto de curso da
        instituição.
      \end{enumerate}
    \end{minipage}
  }
\end{itemize}

\section{Hist\'{o}rico do Campus}
\label{sec:hist-cmpurt}

O IF Goiano Câmpus Urutaí foi criado pela Lei nº 1.923 de 28 de julho
de 1953, com a denominação de Escola Agrícola de Urutaí - GO,
subordinada a Superintendência do Ensino Agrícola e Veterinário - SEAV
- do Ministério da Agricultura, iniciou suas atividades em março de
1956, nas instalações da antiga Fazenda Modelo, oferecendo o Curso de
Iniciação Agrícola e de Mestria Agrícola.

Em 1964 pelo Decreto nº. 53.558, de 13 de fevereiro, foi alterada a
denominação de Escola Agrícola para Ginásio Agrícola de Urutaí.

Em 1977, conforme portaria n\textsuperscript{\d o} 32, foi autorizado
o funcionamento do Curso Técnico em Agropecuária, em nível de
2\textsuperscript{\d o} Grau, já com a denominação de Escola Agro
técnica Federal de Urutaí.

Em 16 de novembro de 1993, a então Escola Agro técnica Federal de
Urutaí foi constituída sob a forma de Autarquia Federal, mediante a
Lei nº. 8.731, vinculada à Secretaria de Educação Profissional e
Tecnológica - SETEC - do Ministério da Educação – MEC.  Em função de
sua credibilidade junto ao MEC, em 1997, recebeu a incumbência de
implantar uma Unidade de Ensino Descentralizada - UNED - na cidade de
Morrinhos - GO, sendo um projeto de parceria entre União, Estado e
Município.

Em 1999, foi implantado o Curso Superior de Tecnologia em Irrigação e
Drenagem - TID, inaugurando um novo tempo para a evolução histórica do
então CEFET Urutaí, contribuindo para a sua inserção no Ensino
Superior.

Pelo Decreto Presidencial de 16 de agosto de 2002, houve a
transformação e mudança de denominação de Escola Agro técnica Federal
de Urutaí para Centro Federal de Educação Tecnológica de Urutaí -
CEFET.  Posteriormente, com o Decreto n\textsuperscript{\d o}.  5225,
de 1\textsuperscript{\d o} outubro de 2004, o CEFET Urutaí passa a ser
Instituição de Ensino Superior.  Pela Lei n\textsuperscript{\d o}
11.892 de dezembro de 2008, o CEFET Urutaí foi transformado em IF
Goiano - CâmpusUrutaí que tem como missão: \\
\vspace{1cm}\hfill \adjustbox{max totalheight=0.5\textheight}{%
  \begin{minipage}[t]{10cm}
    Oferecer educação profissional e tecnológica, de forma
    indissociável da pesquisa e extensão buscando o padrão de
    excelência na formação integral de profissionais com valores
    éticos e humanos para o mundo do trabalho, contribuindo com o
    desenvolvimento sustentável e a qualidade de vida da sociedade
    (PDI, p.8-9).
  \end{minipage}
}

O IF Goiano -CampusUrutaí tem como característica o compromisso com a
sociedade, fato que vêm se comprovando na medida em que investe na
implantação de cursos que atendem às demandas do mundo globalizado e
da região em que se insere, sempre com a intenção de fomentar a
criação, produção e difusão de novos conhecimentos e tecnologias.

Ressalta-se que na década de 2000, a instituição expandiu sua oferta
em cursos de graduação.  Em 2003, ofertou o Curso Superior de
Tecnologia em Sistemas de Informação, hoje denominado de Curso
Superior de Tecnologia em Análise e Desenvolvimento de Sistemas. Em
2006, ofereceu o Curso Superior de Tecnologia em Alimentos. Em 2007,
houve a oferta de dois novos cursos superiores de Tecnologia: Gestão
Ambiental e Gestão da Tecnologia da Informação.  Esses cursos foram
constituídos a partir da demanda e em conformidade com as legislações
do Curso de Tecnologia.

Ampliando a oferta de cursos, no primeiro semestre de 2008 começou a
ser ofertado o curso de Bacharelado em Agronomia para atender demanda
existente no contexto regional.

Dando continuidade ao seu desenvolvimento e, procurando atender a Lei
n\textsuperscript{\d o} 11.892, de 29 de dezembro de 2008, a qual
instituiu a Rede Federal de Educação Profissional, Científica e
Tecnológica, criando os Institutos Federais de Educação, Ciência e
Tecnologia, que apresenta como uma das suas finalidades a oferta de
educação profissional e tecnológica para formar e qualificar cidadãos
com vistas na atuação profissional nos diversos setores da economia,
com ênfase no desenvolvimento socioeconômico local, regional e
nacional, o IF Goiano - Campus Urutaí ampliou a sua oferta de cursos.

A Legislação supracitada estabelece que 20\% das vagas ofertadas
deverão ser reservadas aos cursos de Licenciatura e Programas
Especiais de formação pedagógica, com vistas à formação de professores
para educação básica, principalmente, nas áreas de Ciências e
Matemática, e para educação profissional; 50\% correspondem à formação
de cursos técnicos de nível médio e 30\% aos cursos de bacharelado,
engenharias, tecnológicos e de pós-graduação (lato sensu e stricto
sensu).

Nessa direção e considerando o contexto regional, foram abertos novos
cursos superiores.  Em 2009, foi criado o curso de Matemática
(Licenciatura); em 2010, Engenharia Agrícola (Bacharelado) e Ciências
Biológicas (Licenciatura); e em 2011 Químicas (licenciatura).

Em decorrência da oferta de novos cursos, houve aumento no número de
alunos nos cursos superiores e considerado aumento no corpo docente e
Técnicos Administrativos.

\section{Hist\'{o}rico do Curso}
\label{sec:historico-curso}

No ano de dois mil e oito foi constituída uma comissão composta pelos
membros: Aníbal Sebastião Alves Filho, Juliana Cristina da Costa
Fernandes, Júlio César Ferreira (Presidente), Ana Carolina Simões
Lamounier F.  dos Santos, Thelma Maria de Moura Moreno e Cláudio
Umberto Melo.  Coube a essa comissão a responsabilidade de conceber o
Projeto Pedagógico do Curso de Licenciatura em Matemática.

O curso iniciou-se no ano de 2009, por meio da Resolução
n\textsuperscript{\d o} 007 de maio de 2008, com disponibilização de
trinta vagas. O ingresso no curso ocorre no início de cada ano com a
organização curricular semestral, sendo que a partir de 2010 foram
oferecidas quarenta vagas.

\section{Justificativa Para a Implanta\c{c}\~{a}o do Curso}
\label{sec:just-impl-curso}

A dinâmica e a velocidade cada vez maior das mudanças sociais,
políticas, econômicas e culturais da sociedade moderna caracterizam o
que se convencionou chamar de `novo milênio'. No passado as mudanças
significativas na vida humana exigiriam no mínimo o tempo
correspondente a uma geração para ocorrer.  Gradativamente passaram a
ser imprevisíveis. Trata-se da `era da incerteza', conforme denominou
Galbraith(1976) ou, ainda, da `era de descontinuidade', como
classificou Drucker (1974).

O sentido de `novo milênio' identifica-se, assim, com as
transformações globais que caracterizam o mundo moderno.
Informatização, comunicação, mundialização e sociedade do conhecimento
são alguns fatores que estão pressionando o status quoda vida atual.
Sobretudo mudanças de valores e crenças individuais e culturais marcam
a sociedade atual.

A educação superior é uma instituição social, estável e duradoura,
concebida a partir de normas e valores da sociedade. É, acima de tudo,
um ideal que se destina, quanto à qualificação profissional e promoção
do desenvolvimento político, econômico, social e cultural.

Para atender a necessidade imposta por essas mudanças, tem-se
observado no setor educacional a preocupação no sentido de formular
políticas públicas, que possa orientar e organizar o funcionamento das
instituições educacionais em todos os níveis de modalidade.

Nesse sentido os Centros Federais de Educação Tecnológica passaram por
transformações na sua estrutura administrativa, didática e
organizacional no ano de 2008, com a adoção do modelo Ifet -
Institutos Federais de Educação, Ciência e Tecnologia.

O Estado de Goiás foi contemplado com dois Ifet's: o Ifet Goiás,
constituído pelo Cefet Goiás e suas Uned's e o Ifet Goiano,
constituídopelo Cefet Urutaí, Uned Morrinhos, Cefet Rio Verde e a
Escola Agrotécnica Federal de Ceres. Este novo modelo possui algumas
especificidades, entre elas, a exigência de ofertar o mínimo de 20\%
das vagas do Ifet para cursos de Licenciatura, preferencialmente,
Biologia, Física, Química e Matemática.

A grande deficiência de profissionais da educação voltados para a
licenciatura em Matemática está demonstrada na Tabela
\ref{tab:inep2003}, que revela o quão necessário para a região da
estrada de ferro e todo o Estado de Goiás a implantação do curso de
Licenciatura em Matemática nesta Instituição.

Pesquisa do INEP de 2003, considerando o número de profissionais do
Magistério da Educação Básica no Estado de Goiás que ministram a
disciplina Matemática, por curso de graduação concluído, segundo a
unidade da federação, dependência administrativa e localização fornece
os seguintes dados:

\begin{table}[!htbp]
  \centering
  \caption{Percentual de funções docentes que atuam no Ensino Médio
    por grau de formação -- Brasil e regiões -- 1991/2002}
  \label{tab:inep2003}
  \adjustbox{max width=1.01\textwidth}{%
    \begin{tabular}{@{}*7c@{}}
      \toprule
      & & \multicolumn{5}{c}{Grau de Formação} \\
      \cmidrule{3-7}
      & & Fundamental & \multicolumn{2}{c}{Médio} & \multicolumn{2}{c}{Superior} \\
      Unidade & & & Com & Sem & Sem & Com \\
      Geográfica & Ano & & Magistério & Magistério & Licenciatura & Licenciatura \\
      \midrule
      \multirow{3}{*}{Brasil} & 1991 & 0.4 & 6.8 & 9.4 & 8.5 & 74,9 \\
      & 1996 & 0.3 & 6.9 & 6.4 & 12.1 & 74.3 \\
      & 2002 & 0.1 & 5.2 & 5.4 & 10.3 & 79.0 \\
      \midrule
      \multirow{3}{*}{Norte} & 1991 & 0.4 & 11.2 & 16.9 & 7.2 & 64.3 \\
      & 1996 & 0.4 & 13.0 & 7.2 & 16.2 & 63.2 \\
      & 2002 & 0.0 & 9.9 & 5.6 & 14.9 & 69.6 \\
      \midrule
      \multirow{3}{*}{Nordeste} & 1991 & 0.8 & 18.8 & 13.0 & 7.8 & 59.7 \\
      & 1996 & 0.6 & 16.6 & 7.8 & 13.9 & 61.0 \\
      & 2002 & 0.1 & 12.2 & 7.7 & 13.3 & 66.7 \\
      \midrule
      \multirow{3}{*}{Sudeste} & 1991 & 0.3 & 2.2 & 8.0 & 8.8 & 80.8 \\
      & 1996 & 0.2 & 2.8 & 5.4 & 11.7 & 80.0 \\
      & 2002 & 0.0 & 1.2 & 3.5 & 8.3 & 87.0 \\
      \midrule
      \multirow{3}{*}{Sul} & 1991 & 0.2 & 2.7 & 6.8 & 8.3 & 82.0 \\
      & 1996 & 0.3 & 2.6 & 6.1 & 10.0 & 80.9 \\
      & 2002 & 0.1 & 2.2 & 5.9 & 10.7 & 81.0 \\
      \midrule
      \multirow{3}{*}{Centro-Oeste} & 1991 & 0.5 & 12.0 & 10.4 & 9.5 & 67.6 \\
      & 1996 & 0.3 & 10.9 & 9.4 & 11.7 & 76.7 \\
      & 2002 & 0.1 & 11.5 & 9.1 & 9.2 & 70.1 \\
      \bottomrule
    \end{tabular}
  }
\end{table}



%%% Local Variables:
%%% mode: latex
%%% TeX-master: "../../main"
%%% End:
